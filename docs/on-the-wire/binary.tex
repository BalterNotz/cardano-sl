\documentclass{article}

\usepackage{amssymb}
\usepackage{amsmath}
\usepackage{appendix}
\usepackage{float}
\usepackage{enumitem}

\floatstyle{boxed}
\restylefloat{figure}

\newlist{record}{description}{3}
\setlist[record]{itemsep=0pt, format=$-$\bfseries}
\newlist{variant}{description}{3}
\setlist[variant]{itemsep=0pt, format=$|$\bfseries}
\newcommand{\N}{\mathbb{N}}

\newcommand{\idsof}[1]{\mathcal{I}\!_#1}

\newcommand{\txs}{\mathcal{T}}
\newcommand{\blocks}{\mathcal{B}}
\newcommand{\bbodies}{\blocks_{\textbf{body}}}
\newcommand{\bheads}{\blocks_{\textbf{head}}}
\newcommand{\txins}{\txs_{\textbf{in}}}

\newcommand{\txids}{\idsof{\txs}}
\newcommand{\blockids}{\idsof{\blocks}}
\newcommand{\agentids}{\idsof{\mathcal{A}}}

\newcommand{\hstype}[1]{\textbf{#1}}
\newcommand{\String}{\hstype{String}}
\newcommand{\Word}[1]{\hstype{Word#1}}


\title{Cardano on-the-wire specification}
\author{Nicholas Clarke}
\begin{document}
\maketitle
\tableofcontents

\section{Introduction}

This document describes the binary serialisation formats used for the Cardano
blockchain.

It proceeds as follows: section \ref{sec:types} describes the core types which
must be serialised for the purposes of Cardano communication. We then detail the
requirements on the binary format in \ref{sec:reqs}. In section \ref{sec:binfmt}
we describe the explicit binary serialisation of the core types satisfying these
requirements.

Appendix \ref{sec:currentfmt} details the current binary format.

\subsection{Notation}

\begin{description}
  \item[Records] We denote records as:
    \begin{record}
      \item[field1] Field description
      \item[field2] Field description
    \end{record}

  \item[Variants] We denote variants as:
    \begin{variant}
      \item[Option1] Description
      \item[Option2] Description
    \end{variant}
\end{description}

Typically we will use capitalised words for variant constructors and lower-case
terms for field names, as per standard Haskell syntax.

\section{Core datatypes}
\label{sec:types}

The types in this section are derived from a combination of the blockchain spec
and Haskell datatypes in the codebase.

\subsection{Primitives}

Primitive types:

\begin{description}
\item [$\Word{8}$] 8-bit word
\item [$\Word{32}$] 32-bit word
\item [$\Word{64}$] 64-bit word
\item [$\String$] Arbitrary string type. We do not draw any distinctions as to
  whether this is implemented as a $\hstype{String}$, $\hstype{ByteString}$ or
  $\hstype{Text}$.
\end{description}

\subsection{Identifiers}

We start with sets of identifiers. These are represented in code as Blake2b-256
hashes.

\begin{itemize}
\item{Transaction identifiers $\txids$}
\item{Agent identifiers $\agentids$}
\item{Block identifiers $\blockids$}
\end{itemize}

\subsection{Transactions}

We have a set of transactions $\txs$.

\begin{figure}[H]
\caption{Transaction input $\in \txins$.}
\label{fig:txin}
\begin{variant}
  \item [Valid] a pair in $\txids\times\Word{32}$.
  \item [Invalid] a pair in $\Word{8}\times\String$.
\end{variant}
\end{figure}

\begin{figure}[H]
  \caption{Transaction $\in\txs$.}
  \label{fig:transaction}
  \begin{record}
    \item[inputs] A list of transaction inputs in $\txins$.
    \item[outputs] A non-empty list of pairs in $\agentids\times\Word{64}$.
  \end{record}
\end{figure}

\subsection{Blocks}

A block consists of a block header and a block body. The block header consists
of verification for the various components in the block body.

\begin{figure}[H]
  \caption{Block body $\in\bbodies$}
  \begin{record}
    \item [txPayload] \hfill
      \begin{record}
        \item [transactions] A list of transactions in $\txs$.
        \item [witnesses] A list of transaction witnesses.
      \end{record}

    \item [sscPayload] \hfill
      \begin{variant}
        \item [Commitments]
        \item [Openings]
        \item [Shares]
        \item [Certificates]
      \end{variant}

    \item [dlgPayload]
    \item [updatePayload]
  \end{record}
  \label{fig:blockbody}
\end{figure}

\section{Requirements on the binary format}
\label{sec:reqs}

\subsection{Cryptographic properties}

\subsection{Dependencies on the binary format}

\section{Binary specification}
\label{sec:binfmt}

\begin{appendices}
  \section{Current binary format}
  \label{sec:currentfmt}

  This section documents the current binary serialisation format, as of
  2018-06-03.
\end{appendices}
\end{document}
